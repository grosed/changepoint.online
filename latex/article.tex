\documentclass[article]{jss}

%% -- LaTeX packages and custom commands ---------------------------------------

%% recommended packages
\usepackage{orcidlink,thumbpdf,lmodern}

%% another package (only for this demo article)
\usepackage{framed}

%% new custom commands
\newcommand{\class}[1]{`\code{#1}'}
\newcommand{\fct}[1]{\code{#1()}}


%% -- Article metainformation (author, title, ...) -----------------------------

%% - \author{} with primary affiliation (and optionally ORCID link)
%% - \Plainauthor{} without affiliations
%% - Separate authors by \And or \AND (in \author) or by comma (in \Plainauthor).
%% - \AND starts a new line, \And does not.

%\author{Alex Fisch\\Lancaster University
%  \And Daniel Grose\\Lancaster University
%  \And Idris A. Eckley\\Lancaster University
%  \AND Paul Fearnhead\\Lancaster University 
%   \And Lawrence Bardwell\\Lancaster University}
%\Plainauthor{Alex Fisch, Daniel Grose, Idris A. Eckley, Paul Fearnhead, Lawrence Bardwell}




\author{Idris A. Eckley\\Lancaster University
  \And Paul Fearnhead\\Lancaster University
  \And Daniel Grose\\Lancaster University
  \AND Gaetano Romano\\Lancaster University
  \And Kes Ward\\Lancaster University}
\Plainauthor{Idris A. Eckley, Paul Fearnhead, Daniel Grose, Gaetano Romano, Kes Ward}


%% - \title{} in title case
%% - \Plaintitle{} without LaTeX markup (if any)
%% - \Shorttitle{} with LaTeX markup (if any), used as running title

\title{\pkg{changepoint\_online} : A \proglang{Python} Package for Online Changepoint Detection }
\Plaintitle{\pkg{changepoint\_online} : A Python Package for Online Changepoint Detection}
\Shorttitle{Online Changepoint Detection}




%% - \Abstract{} almost as usual
\Abstract{
}

%% - \Keywords{} with LaTeX markup, at least one required
%% - \Plainkeywords{} without LaTeX markup (if necessary)
%% - Should be comma-separated and in sentence case.
\Keywords{online,changepoint}
\Plainkeywords{online,changepoint}

%% - \Address{} of at least one author
%% - May contain multiple affiliations for each author
%%   (in extra lines, separated by \emph{and}\\).
%% - May contain multiple authors for the same affiliation
%%   (in the same first line, separated by comma).
\Address{
  Achim Zeileis\\
  Journal of Statistical Software\\
  \emph{and}\\
  Department of Statistics\\
  Faculty of Economics and Statistics\\
  Universit\"at Innsbruck\\
  Universit\"atsstr.~15\\
  6020 Innsbruck, Austria\\
  E-mail: \email{Achim.Zeileis@R-project.org}\\
  URL: \url{https://www.zeileis.org/}
}

\begin{document}


\section{Introduction} \label{section:introduction}



\section{Theory} \label{section:theory}



\section{Examples} \label{section:examples}
%
%
\subsection{Gaussian}
Generate a simulated time series
%
\begin{CodeChunk}
\begin{CodeInput}
import numpy as np
np.random.seed(0)
Y = np.concatenate((np.random.normal(loc=0.0, scale=1.0, size=5000),
                    np.random.normal(loc=10.0, scale=1.0, size=5000)
\end{CodeInput}
\end{CodeChunk}
%
and create a detector based on a Gaussian distribution
%
\begin{CodeChunk}
\begin{CodeInput}
detector = Focus(Gaussian(loc=0))                
\end{CodeInput}
\end{CodeChunk}
%
The detector is now sequentially updated as new data arrives and a changepoint reported if the preset threshold is exceeded.
%
\begin{CodeChunk}
\begin{CodeInput}
threshold = 10.0
for y in Y:
    detector.update(y)
    if detector.statistic() >= threshold:
        break
detector.changepoint()
\end{CodeInput}
\begin{CodeOutput}
{'stopping_time': 5001, 'changepoint': 5000}
\end{CodeOutput}
\end{CodeChunk}
%

Similar methods are available for Poisson, Gamma, and Bernoulli distributions.
%
%
\subsection{Poisson}
%
\begin{CodeChunk}
\begin{CodeInput}
np.random.seed(0)
Y = np.concatenate((np.random.poisson(lam=4.0, size=5000),
                    np.random.poisson(lam=4.2, size=5000)))
detector = Focus(Poisson())
threshold = 10.0
for y in Y:
    detector.update(y)
    if detector.statistic() >= threshold:
        break
detector.changepoint()
\end{CodeInput}
\begin{CodeOutput}
{'stopping_time': 6488, 'changepoint': 4973}
\end{CodeOutput}
\end{CodeChunk}
%
%
\subsection{Gamma}
%
\begin{CodeChunk}
\begin{CodeInput}
np.random.seed(0)
Y = np.concatenate((np.random.gamma(4.0, scale=3.0, size=5000),
                    np.random.gamma(4.0, scale=6.0, size=5000)))
detector = Focus(Gamma(scale=3.0, shape=4.0))
threshold = 10.0
for y in Y:
    detector.update(y)
    if detector.statistic() >= threshold:
        break
detector.changepoint()
\end{CodeInput}
\begin{CodeOutput}
{'stopping_time': 5010, 'changepoint': 5002}
\end{CodeOutput}
\end{CodeChunk}
%
%
\subsection{Bernoulli}
%
\begin{CodeChunk}
\begin{CodeInput}
np.random.seed(0)
Y = np.concatenate((np.random.binomial(n=1, p=.4, size=5000),
                    np.random.binomial(n=1, p=.8, size=5000)))
detector = Focus(Bernoulli())
threshold = 15.0
for y in Y:
    detector.update(y)
    if detector.statistic() >= threshold:
        break
detector.changepoint()
\end{CodeInput}
\begin{CodeOutput}
{'stopping_time': 5024, 'changepoint': 4999}
\end{CodeOutput}
\end{CodeChunk}





%% -- Summary/conclusions/discussion -------------------------------------------

\section{Summary and discussion} \label{section:summary}




%% -- Optional special unnumbered sections -------------------------------------

\section{Applications} \label{section:applications}

\section{Conclusions} \label{section:conslcusions}

\section{Acknowledgments} \label{section:acknowledgments}


\bibliography{refs}




\newpage

\begin{appendix}

\section*{Technical Details} \label{appendix:technical-details}


\section*{Related Publications} \label{appendix:technical-details}
First appearance \cite{fast-online-changepoint-detection-via-functional-pruning-cusum-statistics} \\
First appearance \cite{a-constant-per-iteration-likelihood-ratio-test-for-online-changepoint-detection-for-exponential-family-models} \\
First appearance \cite{a-log-linear-nonparametric-online-changepoint-detection-algorithm-based-on-functional-pruning}

Second appearance \citep{fast-online-changepoint-detection-via-functional-pruning-cusum-statistics} \\
Second appearance \citep{a-constant-per-iteration-likelihood-ratio-test-for-online-changepoint-detection-for-exponential-family-models} \\
Second appearance \citep{a-log-linear-nonparametric-online-changepoint-detection-algorithm-based-on-functional-pruning}



\end{appendix}

%% -----------------------------------------------------------------------------


\end{document}
